% Options for packages loaded elsewhere
\PassOptionsToPackage{unicode}{hyperref}
\PassOptionsToPackage{hyphens}{url}
\PassOptionsToPackage{dvipsnames,svgnames,x11names}{xcolor}
%
\documentclass[
  letterpaper,
  DIV=11,
  numbers=noendperiod]{scrartcl}

\usepackage{amsmath,amssymb}
\usepackage{iftex}
\ifPDFTeX
  \usepackage[T1]{fontenc}
  \usepackage[utf8]{inputenc}
  \usepackage{textcomp} % provide euro and other symbols
\else % if luatex or xetex
  \usepackage{unicode-math}
  \defaultfontfeatures{Scale=MatchLowercase}
  \defaultfontfeatures[\rmfamily]{Ligatures=TeX,Scale=1}
\fi
\usepackage{lmodern}
\ifPDFTeX\else  
    % xetex/luatex font selection
\fi
% Use upquote if available, for straight quotes in verbatim environments
\IfFileExists{upquote.sty}{\usepackage{upquote}}{}
\IfFileExists{microtype.sty}{% use microtype if available
  \usepackage[]{microtype}
  \UseMicrotypeSet[protrusion]{basicmath} % disable protrusion for tt fonts
}{}
\makeatletter
\@ifundefined{KOMAClassName}{% if non-KOMA class
  \IfFileExists{parskip.sty}{%
    \usepackage{parskip}
  }{% else
    \setlength{\parindent}{0pt}
    \setlength{\parskip}{6pt plus 2pt minus 1pt}}
}{% if KOMA class
  \KOMAoptions{parskip=half}}
\makeatother
\usepackage{xcolor}
\setlength{\emergencystretch}{3em} % prevent overfull lines
\setcounter{secnumdepth}{-\maxdimen} % remove section numbering
% Make \paragraph and \subparagraph free-standing
\makeatletter
\ifx\paragraph\undefined\else
  \let\oldparagraph\paragraph
  \renewcommand{\paragraph}{
    \@ifstar
      \xxxParagraphStar
      \xxxParagraphNoStar
  }
  \newcommand{\xxxParagraphStar}[1]{\oldparagraph*{#1}\mbox{}}
  \newcommand{\xxxParagraphNoStar}[1]{\oldparagraph{#1}\mbox{}}
\fi
\ifx\subparagraph\undefined\else
  \let\oldsubparagraph\subparagraph
  \renewcommand{\subparagraph}{
    \@ifstar
      \xxxSubParagraphStar
      \xxxSubParagraphNoStar
  }
  \newcommand{\xxxSubParagraphStar}[1]{\oldsubparagraph*{#1}\mbox{}}
  \newcommand{\xxxSubParagraphNoStar}[1]{\oldsubparagraph{#1}\mbox{}}
\fi
\makeatother

\usepackage{color}
\usepackage{fancyvrb}
\newcommand{\VerbBar}{|}
\newcommand{\VERB}{\Verb[commandchars=\\\{\}]}
\DefineVerbatimEnvironment{Highlighting}{Verbatim}{commandchars=\\\{\}}
% Add ',fontsize=\small' for more characters per line
\usepackage{framed}
\definecolor{shadecolor}{RGB}{241,243,245}
\newenvironment{Shaded}{\begin{snugshade}}{\end{snugshade}}
\newcommand{\AlertTok}[1]{\textcolor[rgb]{0.68,0.00,0.00}{#1}}
\newcommand{\AnnotationTok}[1]{\textcolor[rgb]{0.37,0.37,0.37}{#1}}
\newcommand{\AttributeTok}[1]{\textcolor[rgb]{0.40,0.45,0.13}{#1}}
\newcommand{\BaseNTok}[1]{\textcolor[rgb]{0.68,0.00,0.00}{#1}}
\newcommand{\BuiltInTok}[1]{\textcolor[rgb]{0.00,0.23,0.31}{#1}}
\newcommand{\CharTok}[1]{\textcolor[rgb]{0.13,0.47,0.30}{#1}}
\newcommand{\CommentTok}[1]{\textcolor[rgb]{0.37,0.37,0.37}{#1}}
\newcommand{\CommentVarTok}[1]{\textcolor[rgb]{0.37,0.37,0.37}{\textit{#1}}}
\newcommand{\ConstantTok}[1]{\textcolor[rgb]{0.56,0.35,0.01}{#1}}
\newcommand{\ControlFlowTok}[1]{\textcolor[rgb]{0.00,0.23,0.31}{\textbf{#1}}}
\newcommand{\DataTypeTok}[1]{\textcolor[rgb]{0.68,0.00,0.00}{#1}}
\newcommand{\DecValTok}[1]{\textcolor[rgb]{0.68,0.00,0.00}{#1}}
\newcommand{\DocumentationTok}[1]{\textcolor[rgb]{0.37,0.37,0.37}{\textit{#1}}}
\newcommand{\ErrorTok}[1]{\textcolor[rgb]{0.68,0.00,0.00}{#1}}
\newcommand{\ExtensionTok}[1]{\textcolor[rgb]{0.00,0.23,0.31}{#1}}
\newcommand{\FloatTok}[1]{\textcolor[rgb]{0.68,0.00,0.00}{#1}}
\newcommand{\FunctionTok}[1]{\textcolor[rgb]{0.28,0.35,0.67}{#1}}
\newcommand{\ImportTok}[1]{\textcolor[rgb]{0.00,0.46,0.62}{#1}}
\newcommand{\InformationTok}[1]{\textcolor[rgb]{0.37,0.37,0.37}{#1}}
\newcommand{\KeywordTok}[1]{\textcolor[rgb]{0.00,0.23,0.31}{\textbf{#1}}}
\newcommand{\NormalTok}[1]{\textcolor[rgb]{0.00,0.23,0.31}{#1}}
\newcommand{\OperatorTok}[1]{\textcolor[rgb]{0.37,0.37,0.37}{#1}}
\newcommand{\OtherTok}[1]{\textcolor[rgb]{0.00,0.23,0.31}{#1}}
\newcommand{\PreprocessorTok}[1]{\textcolor[rgb]{0.68,0.00,0.00}{#1}}
\newcommand{\RegionMarkerTok}[1]{\textcolor[rgb]{0.00,0.23,0.31}{#1}}
\newcommand{\SpecialCharTok}[1]{\textcolor[rgb]{0.37,0.37,0.37}{#1}}
\newcommand{\SpecialStringTok}[1]{\textcolor[rgb]{0.13,0.47,0.30}{#1}}
\newcommand{\StringTok}[1]{\textcolor[rgb]{0.13,0.47,0.30}{#1}}
\newcommand{\VariableTok}[1]{\textcolor[rgb]{0.07,0.07,0.07}{#1}}
\newcommand{\VerbatimStringTok}[1]{\textcolor[rgb]{0.13,0.47,0.30}{#1}}
\newcommand{\WarningTok}[1]{\textcolor[rgb]{0.37,0.37,0.37}{\textit{#1}}}

\providecommand{\tightlist}{%
  \setlength{\itemsep}{0pt}\setlength{\parskip}{0pt}}\usepackage{longtable,booktabs,array}
\usepackage{calc} % for calculating minipage widths
% Correct order of tables after \paragraph or \subparagraph
\usepackage{etoolbox}
\makeatletter
\patchcmd\longtable{\par}{\if@noskipsec\mbox{}\fi\par}{}{}
\makeatother
% Allow footnotes in longtable head/foot
\IfFileExists{footnotehyper.sty}{\usepackage{footnotehyper}}{\usepackage{footnote}}
\makesavenoteenv{longtable}
\usepackage{graphicx}
\makeatletter
\newsavebox\pandoc@box
\newcommand*\pandocbounded[1]{% scales image to fit in text height/width
  \sbox\pandoc@box{#1}%
  \Gscale@div\@tempa{\textheight}{\dimexpr\ht\pandoc@box+\dp\pandoc@box\relax}%
  \Gscale@div\@tempb{\linewidth}{\wd\pandoc@box}%
  \ifdim\@tempb\p@<\@tempa\p@\let\@tempa\@tempb\fi% select the smaller of both
  \ifdim\@tempa\p@<\p@\scalebox{\@tempa}{\usebox\pandoc@box}%
  \else\usebox{\pandoc@box}%
  \fi%
}
% Set default figure placement to htbp
\def\fps@figure{htbp}
\makeatother

\KOMAoption{captions}{tableheading}
\makeatletter
\@ifpackageloaded{caption}{}{\usepackage{caption}}
\AtBeginDocument{%
\ifdefined\contentsname
  \renewcommand*\contentsname{Table of contents}
\else
  \newcommand\contentsname{Table of contents}
\fi
\ifdefined\listfigurename
  \renewcommand*\listfigurename{List of Figures}
\else
  \newcommand\listfigurename{List of Figures}
\fi
\ifdefined\listtablename
  \renewcommand*\listtablename{List of Tables}
\else
  \newcommand\listtablename{List of Tables}
\fi
\ifdefined\figurename
  \renewcommand*\figurename{Figure}
\else
  \newcommand\figurename{Figure}
\fi
\ifdefined\tablename
  \renewcommand*\tablename{Table}
\else
  \newcommand\tablename{Table}
\fi
}
\@ifpackageloaded{float}{}{\usepackage{float}}
\floatstyle{ruled}
\@ifundefined{c@chapter}{\newfloat{codelisting}{h}{lop}}{\newfloat{codelisting}{h}{lop}[chapter]}
\floatname{codelisting}{Listing}
\newcommand*\listoflistings{\listof{codelisting}{List of Listings}}
\makeatother
\makeatletter
\makeatother
\makeatletter
\@ifpackageloaded{caption}{}{\usepackage{caption}}
\@ifpackageloaded{subcaption}{}{\usepackage{subcaption}}
\makeatother

\usepackage{bookmark}

\IfFileExists{xurl.sty}{\usepackage{xurl}}{} % add URL line breaks if available
\urlstyle{same} % disable monospaced font for URLs
\hypersetup{
  pdftitle={Targeted Learning Statistical Analysis Plan (TL-SAP)},
  pdfauthor={Andrew N. Mertens},
  colorlinks=true,
  linkcolor={blue},
  filecolor={Maroon},
  citecolor={Blue},
  urlcolor={Blue},
  pdfcreator={LaTeX via pandoc}}


\title{Targeted Learning Statistical Analysis Plan (TL-SAP)}
\usepackage{etoolbox}
\makeatletter
\providecommand{\subtitle}[1]{% add subtitle to \maketitle
  \apptocmd{\@title}{\par {\large #1 \par}}{}{}
}
\makeatother
\subtitle{Simulation Study: HCV Treatment and Risk of Acute Kidney
Injury (AKI)}
\author{Andrew N. Mertens}
\date{}

\begin{document}
\maketitle


\section{1. Background and Objectives}\label{background-and-objectives}

This TL-SAP specifies the causal estimand, statistical estimand,
modeling strategy, diagnostics, and evaluation plan for a simulated
real‑world‑data (RWD) safety analysis of the association between
sofosbuvir-containing (SOF) vs non‑SOF direct‑acting antiviral (DAA)
regimens and acute kidney injury (AKI) in individuals with chronic
hepatitis C (HCV).\\
The analysis is designed to emulate a clean‑room staging framework.

The purpose of the simulation is to evaluate how targeted
learning---specifically, TMLE with Super Learner---performs relative to
standard approaches (e.g., Cox, IPTW, GEE) under realistic
data‑generating mechanisms that include measured confounding,
informative censoring, and potential treatment switching.

\section{2. Causal Roadmap}\label{causal-roadmap}

\subsection{2.1 Causal Question}\label{causal-question}

\begin{quote}
\emph{Among adults eligible for HCV DAA treatment, what is the
difference in the risk of AKI by time ( t\^{}* ) if everyone were
treated with SOF-containing vs non‑SOF regimens?}
\end{quote}

\subsection{2.2 Population (P)}\label{population-p}

Simulated representation of an HCV RWD cohort, modeled after a U.S.
claims‑based standing cohort. Each simulated individual contains:

\begin{itemize}
\tightlist
\item
  Baseline covariates ( W ) (demographics, CKD, diabetes, CHF,
  medication use, etc.)
\item
  Treatment regimen ( A\_0 \in \{ \text{SOF}, \text{non‑SOF} \} )
\item
  Time‑varying censoring and AKI outcomes.
\end{itemize}

\subsection{2.3 Intervention / Treatment Strategies
(I)}\label{intervention-treatment-strategies-i}

Static interventions:

\begin{enumerate}
\def\labelenumi{\arabic{enumi}.}
\tightlist
\item
  \textbf{SOF strategy:} Set ( A\_0 = 1 ) for all.
\item
  \textbf{non‑SOF strategy:} Set ( A\_0 = 0 ) for all.
\end{enumerate}

\subsection{2.4 Outcome (O)}\label{outcome-o}

\begin{itemize}
\tightlist
\item
  ( Y\_t = 1 ) if an AKI event occurs by time ( t ); otherwise 0.
\item
  Final outcome: ( Y\_\{t\^{}*\} ) at predetermined time points (e.g.,
  90 days, 180 days).
\end{itemize}

\subsection{2.5 Intercurrent Events}\label{intercurrent-events}

\begin{itemize}
\tightlist
\item
  Treatment discontinuation and switching
\item
  Death
\item
  Loss to follow‑up\\
  These appear in the simulated data as censoring nodes ( C\_t ).
\end{itemize}

\subsection{2.6 Causal Estimand (Target
Parameter)}\label{causal-estimand-target-parameter}

Risk difference at time ( t\^{}* ):

{[} \psi(t\^{}*) = \mathbb{E}{[}Y\_\{t\textsuperscript{*\}}\{A=1\}{]} -
\mathbb{E}{[}Y\_\{t\textsuperscript{*\}}\{A=0\}{]} {]}

Optionally: risk ratio.

Assumptions: consistency, conditional exchangeability, positivity, and
correct specification of either Q or g (via TMLE double robustness).

\section{3. Statistical Estimand}\label{statistical-estimand}

The statistical estimand corresponding to the causal estimand is the
contrast between predicted AKI risks at ( t\^{}* ) under two
intervention-specific longitudinal data distributions constructed using
the estimated outcome regression ( \hat{Q} ) and treatment/censoring
mechanism ( \hat{g} ).

\section{4. Data Structure and Nodes}\label{data-structure-and-nodes}

For each individual ( i ) and discrete time ( t = 1, \ldots, T ):

\begin{itemize}
\tightlist
\item
  ( W\_i ): baseline covariates
\item
  ( A\_\{i0\} ): baseline DAA class (SOF vs non‑SOF)
\item
  ( C\_\{it\} ): censoring indicator at time t
\item
  ( Y\_\{it\} ): AKI event indicator at time t
\end{itemize}

The simulation dataset contains a known ground‑truth data‑generating
process for benchmarking bias, RMSE, and coverage.

\section{5. Estimation Strategy (TMLE)}\label{estimation-strategy-tmle}

\subsection{5.1 Overview}\label{overview}

We will use longitudinal TMLE for survival/competing‑risk‑style
outcomes. Estimation will be performed separately under each
intervention.

\subsection{\texorpdfstring{5.2 Initial Outcome Regression ( \hat{Q}
)}{5.2 Initial Outcome Regression (  )}}\label{initial-outcome-regression}

Super Learner library (modifiable):

\begin{itemize}
\tightlist
\item
  GLM
\item
  GAM
\item
  Random Forest
\item
  XGBoost
\item
  HAL (optional)
\item
  SL.mean as a baseline fallback
\end{itemize}

Cross‑validated risk: Bernoulli log‑likelihood.

\subsection{\texorpdfstring{5.3 Treatment and Censoring Mechanisms (
\hat{g}
)}{5.3 Treatment and Censoring Mechanisms (  )}}\label{treatment-and-censoring-mechanisms}

\begin{itemize}
\tightlist
\item
  Treatment model: ( P(A\_0 \mid W) )
\item
  Censoring model: ( P(C\_t = 1 \mid \bar\{L\}\_t, A\_0, W) )
\end{itemize}

SL library similar to Q.

\subsection{5.4 Targeting Step}\label{targeting-step}

For each ( t ), update ( \hat{Q} ) using the fluctuation submodel:

{[} \text{logit}(\hat{Q}\^{}*) = \text{logit}(\hat Q) + \epsilon H\_t
{]}

where ( H\_t ) is the clever covariate derived from ( \hat g ).

\subsection{5.5 Truncation and
Diagnostics}\label{truncation-and-diagnostics}

\begin{itemize}
\tightlist
\item
  Truncate estimated g-values at {[}0.01, 0.99{]}
\item
  Report effective sample size under weights
\item
  Report positivity violations
\item
  Influence‑curve based standard errors and 95\% CIs
\end{itemize}

\section{6. Simulation Design}\label{simulation-design}

\subsection{6.1 DGP Scenarios}\label{dgp-scenarios}

At minimum:

\begin{enumerate}
\def\labelenumi{\arabic{enumi}.}
\tightlist
\item
  \textbf{Unconfounded scenario} (benchmark)
\item
  \textbf{Baseline confounding only}
\item
  \textbf{Time-varying confounding + informative censoring}
\item
  \textbf{Treatment switching scenario}
\item
  \textbf{Positivity‑stress scenario} (rare SOF or rare non‑SOF)
\end{enumerate}

\subsection{6.2 Sample Size}\label{sample-size}

\begin{itemize}
\tightlist
\item
  Default: ( n = 50\{,\}000 )
\item
  500 simulation replicates per scenario
\end{itemize}

\subsection{6.3 Evaluation Metrics}\label{evaluation-metrics}

\begin{itemize}
\tightlist
\item
  Bias: ( \hat{\psi} - \psi\_\{\text{true}\} )
\item
  RMSE
\item
  Empirical SE vs IC-based SE
\item
  Coverage of 95\% CIs
\item
  TMLE vs IPTW vs unadjusted vs Cox
\end{itemize}

\section{7. Software and
Reproducibility}\label{software-and-reproducibility}

\begin{itemize}
\tightlist
\item
  Implemented in \textbf{R}
\item
  Packages: \texttt{ltmle}, \texttt{lmtp}, \texttt{SuperLearner},
  \texttt{data.table}, \texttt{simcausal}
\item
  Seed fixed for reproducibility
\item
  All code, DGP, and results archived
\end{itemize}

\section{8. Reporting}\label{reporting}

For each scenario:

\begin{itemize}
\tightlist
\item
  Risk curves under SOF and non‑SOF
\item
  RD/RR at each ( t\^{}* )
\item
  Convergence + diagnostics
\item
  TMLE vs comparator estimators
\end{itemize}

\section{9. Deviations From Plan}\label{deviations-from-plan}

All deviations must be documented, justified, and version-controlled, as
in clean‑room decision logs.

\section{Appendix: Simulation
Pseudocode}\label{appendix-simulation-pseudocode}

\begin{Shaded}
\begin{Highlighting}[]
\NormalTok{W }\SpecialCharTok{\textasciitilde{}}\NormalTok{ multivariate normal}
\NormalTok{A0 }\SpecialCharTok{\textasciitilde{}} \FunctionTok{Bernoulli}\NormalTok{(}\FunctionTok{expit}\NormalTok{(WβA))}
\ControlFlowTok{for}\NormalTok{ t }\ControlFlowTok{in} \DecValTok{1}\SpecialCharTok{:}\NormalTok{T}\SpecialCharTok{:}
\NormalTok{    C\_t }\SpecialCharTok{\textasciitilde{}} \FunctionTok{Bernoulli}\NormalTok{(}\FunctionTok{expit}\NormalTok{(WβC }\SpecialCharTok{+}\NormalTok{ A0γC))}
    \ControlFlowTok{if}\NormalTok{ C\_t }\SpecialCharTok{==} \DecValTok{1} \ControlFlowTok{break}
\NormalTok{    Y\_t }\SpecialCharTok{\textasciitilde{}} \FunctionTok{Bernoulli}\NormalTok{(}\FunctionTok{hazard}\NormalTok{(W, A0))}
\end{Highlighting}
\end{Shaded}





\end{document}
